\documentclass{beamer}
\usepackage{graphicx} % Required for inserting images
\usepackage{amsmath}
\usepackage[most]{tcolorbox}
\usepackage{lmodern}
\usepackage{mathabx}

\usetheme{Madrid} % 可選其他主題:e.g., Warsaw, Berkeley, etc.
\usecolortheme{default}
\setbeamertemplate{caption}[numbered]% Number float-like environments
% customize the caption
\setbeamerfont{caption}{size=\footnotesize}
% \setbeamercolor{caption}{fg=blue}
% \setbeamercolor{caption name}{fg=red}

\title{Applications of Differentiation}
\subtitle{section4.2: The Mean Value Theorem}
\author{Hsu Chun-Wei}
\date{June 2025}

\begin{document}

\maketitle

%=====================================================================
\begin{frame}{Rolle's Theorem}
\begin{tcolorbox}[colframe=red!80!black, colback=white, title=\textbf{Rolle's Theorem}]
Let $f$ be a function that satisfies the following three hypotheses:
\begin{enumerate}
  \item $f$ is continuous on the closed interval $[a, b]$.
  \item $f$ is differentiable on the open interval $(a, b)$.
  \item $f(a) = f(b)$
\end{enumerate}
Then there is a number $c$ in $(a, b)$ such that $f'(c) = 0$.
\end{tcolorbox}
\begin{figure}[ht]
    \centering
    \includegraphics[width=\linewidth]{figures/section4.2/CalculusT1.png}
\end{figure}
\end{frame}

%=====================================================================
\begin{frame}{Mean Value Theorem}
\begin{tcolorbox}[colframe=red!80!black, colback=white, title=\textbf{The Mean Value Theorem}]
Let $f$ be a function that satisfies the following hypotheses:
\begin{enumerate}
  \item $f$ is continuous on the closed interval $[a, b]$.
  \item $f$ is differentiable on the open interval $(a, b)$.
\end{enumerate}
Then there is a number $c$ in $(a, b)$ such that
\[
f'(c) = \frac{f(b) - f(a)}{b - a}
\]
\end{tcolorbox}
\begin{figure}[ht]
    \centering
    \includegraphics[width=0.7\linewidth]{figures/section4.2/CalculusT2.png}
\end{figure}
\end{frame}
%=====================================================================
\begin{frame}{Proof of the Mean Value Theorem}
\textbf{\textcolor{red}{PROOF}} \quad We apply Rolle’s Theorem to a new function $h$ defined as the difference between $f$ and the function whose graph is the secant line $AB$. Using Equation 3 and the point-slope equation of a line, we see that the equation of the line $AB$ can be written as
\(
y - f(a) = \frac{f(b) - f(a)}{b - a}(x - a)
\)
\begin{columns}[C]
% create the column with the first image, that occupies
% half of the slide
    \begin{column}{.4\textwidth}
    \begin{figure}
        \label{between 2 and 3}
        \centering
        \includegraphics[width=\textwidth]{figures/section4.2/CalculusT3.png}
    \end{figure}      
    \end{column}
% create the column with the second image, that also
% occupies half of the slide
    \begin{column}{.6\textwidth}
    So, as shown in the Figure,
    \[
    h(x) = f(x) - f(a) - \frac{f(b) - f(a)}{b - a}(x - a)
    \]
    \end{column}
\end{columns}

\end{frame}

%=====================================================================
\begin{frame}{Proof of the Mean Value Theorem}
First, we must verify that $h$ satisfies the three hypotheses of Rolle’s Theorem.
\begin{enumerate}
  \item The function $h$ is continuous on $[a, b]$ because it is the sum of $f$ and a first-degree polynomial, both of which are continuous.
  \item The function $h$ is differentiable on $(a, b)$ because both $f$ and the first-degree polynomial are differentiable. In fact, we can compute $h'$ directly from Equation 4:
  \[
  h'(x) = f'(x) - \frac{f(b) - f(a)}{b - a}
  \]
\end{enumerate}

(Note that $f(a)$ and $[f(b) - f(a)] / (b - a)$ are constants.)
\end{frame}
%=====================================================================
\begin{frame}{Proof of the Mean Value Theorem}

\begin{enumerate}
  \setcounter{enumi}{2}
  \item 
  \[
  h(a) = f(a) - f(a) - \frac{f(b) - f(a)}{b - a}(a - a) = 0
  \]
  \[
  h(b) = f(b) - f(a) - \frac{f(b) - f(a)}{b - a}(b - a)
  \]
  \[
  \phantom{h(b)} = f(b) - f(a) - [f(b) - f(a)] = 0
  \]
  Therefore $h(a) = h(b)$.
\end{enumerate}

Since $h$ satisfies all the hypotheses of Rolle’s Theorem, that theorem says there is a number $c$ in $(a, b)$ such that $h'(c) = 0$. Therefore
\[
0 = h'(c) = f'(c) - \frac{f(b) - f(a)}{b - a}
\]
and so
\[
f'(c) = \frac{f(b) - f(a)}{b - a}
\]

\end{frame}

%=====================================================================
\begin{frame}{Theorem: Zero Derivative Implies Constant Function}
\begin{tcolorbox}[colframe=red!80!black, colback=white, title=\textbf{Theorem}]
If $f'(x) = 0$ for all $x$ in an interval $(a, b)$, then $f$ is constant on $(a, b)$.
\end{tcolorbox}

\textbf{\textcolor{red}{PROOF}} \quad Let $x_1$ and $x_2$ be any two numbers in $(a, b)$ with $x_1 < x_2$. Since $f$ is differentiable on $(a, b)$, it must be differentiable on $(x_1, x_2)$ and continuous on $[x_1, x_2]$. By applying the Mean Value Theorem to $f$ on the interval $[x_1, x_2]$, we get a number $c$ such that $x_1 < c < x_2$ and

\[
f(x_2) - f(x_1) = f'(c)(x_2 - x_1)
\]

Since $f'(x) = 0$ for all $x$, we have $f'(c) = 0$, and so Equation 6 becomes
\[
f(x_2) - f(x_1) = 0 \quad \text{or} \quad f(x_2) = f(x_1)
\]

Therefore $f$ has the same value at \textit{any} two numbers $x_1$ and $x_2$ in $(a, b)$. This means that $f$ is constant on $(a, b)$.

\end{frame}

%=====================================================================
\begin{frame}{Corollary: Equal Derivatives Implies Constant Difference}
\begin{tcolorbox}[colframe=red!80!black, colback=white, title=\textbf{Corollary}]
If $f'(x) = g'(x)$ for all $x$ in an interval $(a, b)$, then $f - g$ is constant on $(a, b)$; that is, $f(x) = g(x) + c$ where $c$ is a constant.
\end{tcolorbox}

\textbf{\textcolor{red}{PROOF}} \quad Let $F(x) = f(x) - g(x)$. Then
\[
F'(x) = f'(x) - g'(x) = 0
\]
for all $x$ in $(a, b)$. Thus, by the theorem above, $F$ is constant; that is, $f - g$ is constant.
\end{frame}
%=====================================================================
%=====================================================================
%=====================================================================
%=====================================================================
%=====================================================================

\end{document}