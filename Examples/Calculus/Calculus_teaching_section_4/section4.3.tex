\documentclass{beamer}
\usepackage{graphicx} % Required for inserting images
\usepackage{amsmath}
\usepackage[most]{tcolorbox}
\usepackage{lmodern}
\usepackage{mathabx}

\usetheme{Madrid} % 可選其他主題:e.g., Warsaw, Berkeley, etc.
\usecolortheme{default}
\setbeamertemplate{caption}[numbered]% Number float-like environments
% customize the caption
\setbeamerfont{caption}{size=\footnotesize}
% \setbeamercolor{caption}{fg=blue}
% \setbeamercolor{caption name}{fg=red}

\title{Applications of Differentiation}
\subtitle{section4.3: What Derivatives Tell Us about the Shape of a Graph}
\author{Hsu Chun-Wei}
\date{June 2025}

\begin{document}

\maketitle

%=====================================================================
\begin{frame}{Increasing/Decreasing Test}
\begin{tcolorbox}[colframe=red!80!black, colback=white, title=\textbf{Increasing/Decreasing Test}]
\begin{enumerate}[(a)]
  \item If $f'(x) > 0$ on an interval, then $f$ is increasing on that interval.
  \item If $f'(x) < 0$ on an interval, then $f$ is decreasing on that interval.
\end{enumerate}
\end{tcolorbox}

\textbf{\textcolor{red}{PROOF}} \quad (a) Let $x_1$ and $x_2$ be any two numbers in the interval with $x_1 < x_2$. we have to show that $f(x_1) < f(x_2)$.

Because we are given that $f'(x) > 0$, we know that $f$ is differentiable on $[x_1, x_2]$. So, by the Mean Value Theorem, there is a number $c$ between $x_1$ and $x_2$ such that
\[
f(x_2) - f(x_1) = f'(c)(x_2 - x_1)
\]

Thus, the right side is positive, and so
\[
f(x_2) - f(x_1) > 0 \quad \text{or} \quad f(x_1) < f(x_2)
\]

This shows that $f$ is increasing. Part (b) is proved similarly.
\end{frame}
%=====================================================================
\begin{frame}{The First Derivative Test}
\begin{tcolorbox}[colframe=red!80!black, colback=white, title=\textbf{The First Derivative Test}]
Suppose that $c$ is a critical number of a continuous function $f$.
\begin{enumerate}[(a)]
  \item If $f'$ changes from positive to negative at $c$, then $f$ has a local maximum at $c$.
  \item If $f'$ changes from negative to positive at $c$, then $f$ has a local minimum at $c$.
  \item If $f'$ is positive to the left and right of $c$, or negative to the left and right of $c$, then $f$ has no local maximum or minimum at $c$.
\end{enumerate}
\end{tcolorbox}
\begin{figure}[ht]
    \centering
    \includegraphics[width=\linewidth]{figures/section4.3/CalculusT1.png}
\end{figure}
\end{frame}
%=====================================================================
\begin{frame}{Concavity Test}
\begin{tcolorbox}[colframe=red!80!black, colback=white, title=\textbf{Definition}]
If the graph of $f$ lies above all of its tangents on an interval $I$, then $f$ is called \textbf{concave upward} on $I$. If the graph of $f$ lies below all of its tangents on $I$, then $f$ is called \textbf{concave downward} on $I$.
\end{tcolorbox}
\begin{figure}[ht]
    \centering
    \includegraphics[width=0.8\linewidth]{figures/section4.3/CalculusT2.png}
\end{figure}

\end{frame}
%=====================================================================
\begin{frame}{Concavity Test}
\begin{tcolorbox}[colframe=red!80!black, colback=white, title=\textbf{Concavity Test}]
\begin{enumerate}[(a)]
  \item If $f''(x) > 0$ on an interval $I$, then the graph of $f$ is concave upward on $I$.
  \item If $f''(x) < 0$ on an interval $I$, then the graph of $f$ is concave downward on $I$.
\end{enumerate}
\end{tcolorbox}
\end{frame}
%=====================================================================
\begin{frame}{Definition: Inflection Point}
\begin{tcolorbox}[colframe=red!80!black, colback=white, title=\textbf{Definition}]
A point $P$ on a curve $y = f(x)$ is called an \textbf{inflection point} if $f$ is continuous there and the curve changes from concave upward to concave downward or from concave downward to concave upward at $P$.
\end{tcolorbox}
\end{frame}

%=====================================================================
\begin{frame}{Example: Sketching a Function}
\textbf{\textcolor{blue}{EXAMPLE}} \quad Sketch a possible graph of a function $f$ that satisfies the following conditions:
\begin{itemize}
  \item[(i)] $f'(x) > 0$ on $(-\infty, 1)$, \quad $f'(x) < 0$ on $(1, \infty)$
  \item[(ii)] $f''(x) > 0$ on $(-\infty, -2)$ and $(2, \infty)$, \quad $f''(x) < 0$ on $(-2, 2)$
  \item[(iii)] $\displaystyle \lim_{x \to -\infty} f(x) = -2$, \quad $\displaystyle \lim_{x \to \infty} f(x) = 0$
\end{itemize}

\end{frame}
%=====================================================================
\begin{frame}{Example: Sketching a Function}
\textbf{\textcolor{blue}{SOLUTION}} 
\begin{figure}[ht]
    \centering
    \includegraphics[width=0.5\linewidth]{figures/section4.3/CalculusT3.png}
\end{figure}
\begin{itemize}
    \item Condition (i) tells us that $f$ is increasing on $(-\infty, 1)$ and decreasing on $(1, \infty)$.
    \item Condition (ii) says that $f$ is concave upward on $(-\infty, -2)$ and $(2, \infty)$, and concave downward on $(-2, 2)$. 
    \item condition (iii) tells us that the graph of $f$ has two horizontal asymptotes: $y = -2$ (to the left) and $y = 0$ (to the right).
\end{itemize}
\end{frame}
%=====================================================================
\begin{frame}{The Second Derivative Test}
\begin{tcolorbox}[colframe=red!80!black, colback=white, title=\textbf{The Second Derivative Test}]
Suppose $f''$ is continuous near $c$.
\begin{enumerate}[(a)]
  \item If $f'(c) = 0$ and $f''(c) > 0$, then $f$ has a local minimum at $c$.
  \item If $f'(c) = 0$ and $f''(c) < 0$, then $f$ has a local maximum at $c$.
\end{enumerate}
\end{tcolorbox}
\end{frame}

%=====================================================================
\begin{frame}{Example: Sketching a Function}
\textbf{\textcolor{blue}{EXAMPLE}} \quad Sketch the graph of the function $f(x) = x^{2/3}(6 - x)^{1/3}$.

\vspace{0.5em}
\textbf{\textcolor{blue}{SOLUTION}} \quad First note that the domain of $f$ is $\mathbb{R}$. Calculation of the first two derivatives gives:
\[
f'(x) = \frac{4 - x}{x^{1/3}(6 - x)^{2/3}}, \qquad f''(x) = \frac{-8}{x^{4/3}(6 - x)^{5/3}}
\]

Since $f'(x) = 0$ when $x = 4$ and $f'(x)$ does not exist when $x = 0$ or $x = 6$, the critical numbers are 0, 4, and 6.

\vspace{1em}

\begin{center}
\begin{tabular}{|c|c|c|c|c|c|}
\hline
\textbf{Interval} & $4 - x$ & $x^{1/3}$ & $(6 - x)^{2/3}$ & $f'(x)$ & $f$ \\
\hline
$x < 0$ & $+$ & $-$ & $+$ & $-$ & decreasing on $(-\infty, 0)$ \\
$0 < x < 4$ & $+$ & $+$ & $+$ & $+$ & increasing on $(0, 4)$ \\
$4 < x < 6$ & $-$ & $+$ & $+$ & $-$ & decreasing on $(4, 6)$ \\
$x > 6$ & $-$ & $+$ & $-$ & $-$ & decreasing on $(6, \infty)$ \\
\hline
\end{tabular}
\end{center}
\end{frame}
%=====================================================================
\begin{frame}{Example: Sketching a Function}
    \begin{figure}[ht]
        \centering
        \includegraphics[width=0.8\linewidth]{figures/section4.3/CalculusT4.png}
    \end{figure}
\end{frame}
%=====================================================================
%=====================================================================

\end{document}