\documentclass{beamer}
\usepackage{graphicx} % Required for inserting images
\usepackage{amsmath}
\usepackage[most]{tcolorbox}
\usepackage{lmodern}
\usepackage{mathabx}

\usetheme{Madrid} % 可選其他主題:e.g., Warsaw, Berkeley, etc.
\usecolortheme{default}
\setbeamertemplate{caption}[numbered]% Number float-like environments
% customize the caption
\setbeamerfont{caption}{size=\footnotesize}
% \setbeamercolor{caption}{fg=blue}
% \setbeamercolor{caption name}{fg=red}

\title{Applications of Differentiation}
\subtitle{section4.7: Optimization Problems}
\author{Hsu Chun-Wei}
\date{June 2025}

\begin{document}

\maketitle
%=====================================================================
\begin{frame}{Steps in Solving Optimization Problems}

\begin{tcolorbox}[colframe=purple!80!black, colback=white, title=\textbf{Optimization Problem Strategy}]
\begin{enumerate}
  \item \textbf{Understand the Problem} \\
  Read carefully until the problem is clear. Ask:
  \begin{itemize}
    \item What is the unknown?
    \item What are the given quantities and conditions?
  \end{itemize}

  \item \textbf{Draw a Diagram} \\
  Visualize the problem by drawing and labeling given and required quantities.

  \item \textbf{Introduce Notation} \\
  Assign symbols to the quantity to be optimized (e.g., $Q$), and to other unknowns (e.g., $a$, $b$, $x$, $y$). Use suggestive symbols when possible (e.g., $A$ for area, $h$ for height).

  \item \textbf{Express $Q$ in Terms of Other Variables} \\
  Write $Q$ as a function of the other variables introduced.
\end{enumerate}
\end{tcolorbox}

\end{frame}

%=====================================================================
\begin{frame}{Steps in Solving Optimization Problems}

\begin{tcolorbox}[colframe=purple!80!black, colback=white, title=\textbf{Optimization Problem Strategy}]
\begin{enumerate}
  \setcounter{enumi}{4}

  \item \textbf{Reduce to One Variable} \\
  Use constraints to eliminate extra variables and express $Q$ as $Q = f(x)$.
  \begin{itemize}
    \item Determine the domain of $f$ in the context of the problem.
  \end{itemize}

  \item \textbf{Optimize the Function} \\
  Use calculus methods (from Sections 4.1 and 4.3) to find the absolute maximum or minimum of $f$.
  \begin{itemize}
    \item If $f$ is on a closed interval, use the Closed Interval Method (Section 4.1).
  \end{itemize}
\end{enumerate}
\end{tcolorbox}

\end{frame}
%=====================================================================
\begin{frame}{Example 1: Optimization Problem}

\textbf{Example 1.} Find the point on the parabola $y^2 = 2x$ that is closest to the point $(1, 4)$.
\begin{figure}
    \centering
    \includegraphics[width=0.5\linewidth]{figures/section4.7/CalculusT1.png}
\end{figure}

\end{frame}
%=====================================================================
\begin{frame}{Example 1: Optimization Problem}

\textbf{Solution:}
We previously obtained:
\[
d^2 = f(y) = \left( \frac{1}{2} y^2 - 1 \right)^2 + (y - 4)^2
\]

To minimize the distance, we minimize $d^2$. Differentiate:
\[
f'(y) = 2\left( \frac{1}{2} y^2 - 1 \right)\cdot y + 2(y - 4) = y^3 - 8
\]

Solve $f'(y) = 0$:
\[
y^3 - 8 = 0 \quad \Rightarrow \quad y = 2
\]

\textbf{Check for minimum:} \\
Since $f'(y) < 0$ for $y < 2$ and $f'(y) > 0$ for $y > 2$, $y = 2$ gives a local (and absolute) minimum.

\vspace{0.5em}
\textbf{Find corresponding $x$:}
\[
x = \frac{1}{2} y^2 = \frac{1}{2} (2)^2 = 2
\]

\end{frame}

%=====================================================================
\begin{frame}{Example 2: Largest Rectangle Inscribed in a Semicircle}
\textbf{Example 2.} Find the area of the largest rectangle that can be inscribed as a semicircle of radius r.
\begin{figure}
    \centering
    \includegraphics[width=0.5\linewidth]{figures/section4.7/CalculusT2.png}
\end{figure}
\end{frame}
%=====================================================================
\begin{frame}{Example: Largest Rectangle Inscribed in a Semicircle}

\textbf{Problem.} Find the area of the largest rectangle that can be inscribed in a semicircle of radius $r$.

\vspace{0.5em}
\textbf{Step 1: Understand the Problem} \\
The rectangle is inscribed in the upper half of the circle $x^2 + y^2 = r^2$ with center at the origin. The rectangle has two vertices on the semicircle and two on the $x$-axis.

\vspace{0.5em}
\textbf{Step 2: Draw a Diagram} \\
Let $(x, y)$ be the vertex in the first quadrant. Then the rectangle has width $2x$ and height $y$.

\vspace{0.5em}
\textbf{Step 3: Introduce Notation} \\
Area of the rectangle:
\[
A = 2x \cdot y
\]

\end{frame}

%=====================================================================
\begin{frame}{Example: Largest Rectangle Inscribed in a Semicircle}

\vspace{0.5em}
\textbf{Step 4–5: Express $A$ as a Function of One Variable} \\
From the circle equation, $x^2 + y^2 = r^2 \Rightarrow y = \sqrt{r^2 - x^2}$. So:
\[
A(x) = 2x \sqrt{r^2 - x^2}
\]

Domain: $0 \le x \le r$

\vspace{0.5em}
\textbf{Step 6: Maximize $A$} \\
Differentiate:
\[
A'(x) = 2\sqrt{r^2 - x^2} + 2x \cdot \frac{-x}{\sqrt{r^2 - x^2}} = \frac{2(r^2 - 2x^2)}{\sqrt{r^2 - x^2}}
\]

Set $A'(x) = 0$:
\[
r^2 - 2x^2 = 0 \Rightarrow x = \frac{r}{\sqrt{2}}
\]

Substitute back:
\[
A = 2x \cdot \sqrt{r^2 - x^2}
= 2 \cdot \frac{r}{\sqrt{2}} \cdot \sqrt{r^2 - \frac{r^2}{2}}
= \frac{2r}{\sqrt{2}} \cdot \sqrt{\frac{r^2}{2}} = r^2
\]
\end{frame}

%=====================================================================
%=====================================================================
%=====================================================================
%=====================================================================

\end{document}