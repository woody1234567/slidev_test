\documentclass{beamer}
\usepackage{graphicx} % Required for inserting images
\usepackage{amsmath}
\usepackage[most]{tcolorbox}
\usepackage{lmodern}
\usepackage{mathabx}

\usetheme{Madrid} % 可選其他主題:e.g., Warsaw, Berkeley, etc.
\usecolortheme{default}
\setbeamertemplate{caption}[numbered]% Number float-like environments
% customize the caption
\setbeamerfont{caption}{size=\footnotesize}
% \setbeamercolor{caption}{fg=blue}
% \setbeamercolor{caption name}{fg=red}

\title{Applications of Differentiation}
\subtitle{section4.4: Indeterminate Forms and l’Hospital’s Rule}
\author{Hsu Chun-Wei}
\date{June 2025}

\begin{document}

\maketitle

%=====================================================================
\begin{frame}{Indeterminate forms}
    \begin{itemize}
        \item indeterminate form of type $\frac{0}{0}$
        \begin{equation}
            \lim_{x \rightarrow a} \frac{f(x)}{g(x)}
        \end{equation}
        both $f(x) \rightarrow 0$ and $g(x) \rightarrow 0$ as $x \rightarrow a$ 
        \item indeterminate form of type $\frac{\infty}{\infty}$
        \begin{equation}
            \lim_{x \rightarrow a} \frac{f(x)}{g(x)}
        \end{equation}
        both $f(x) \rightarrow \pm \infty$ and $g(x) \rightarrow \pm \infty$ as $x \rightarrow a$ 
    \end{itemize}
\end{frame}
%=====================================================================
\begin{frame}{L'Hospital's Rule}

Suppose $f$ and $g$ are differentiable and $g'(x) \ne 0$ on an open interval $I$ that contains $a$ (except possibly at $a$). Suppose that
\[
\lim_{x \to a} f(x) = 0 \quad \text{and} \quad \lim_{x \to a} g(x) = 0
\]
or that
\[
\lim_{x \to a} f(x) = \pm \infty \quad \text{and} \quad \lim_{x \to a} g(x) = \pm \infty
\]
(In other words, we have an indeterminate form of type $\frac{0}{0}$ or $\frac{\infty}{\infty}$.) Then

\begin{tcolorbox}[colframe=red!80!black, colback=white, title=\textbf{L'Hospital's Rule}]
\[
\lim_{x \to a} \frac{f(x)}{g(x)} = \lim_{x \to a} \frac{f'(x)}{g'(x)}
\]
\end{tcolorbox}

if the limit on the right side exists (or is $\infty$ or $-\infty$).

\end{frame}

%=====================================================================
\begin{frame}{Intuition Behind L'Hospital's Rule}

\begin{columns}[c]
% create the column with the first image, that occupies
% half of the slide
        \begin{column}{.6\textwidth}
The first graph shows two differentiable functions $f$ and $g$, each of which approaches 0 as $x \to a$. If we zoom in toward the point $(a, 0)$, the graphs start to look almost linear.

If the functions \textit{were} linear, as in the second graph, then their ratio would be:
\[
\frac{m_1(x - a)}{m_2(x - a)} = \frac{m_1}{m_2}
\]

which is the ratio of their derivatives.

\begin{tcolorbox}[colframe=red!80!black, colback=white, title=\textbf{This suggests:}]
\[
\lim_{x \to a} \frac{f(x)}{g(x)} = \lim_{x \to a} \frac{f'(x)}{g'(x)}
\]
\end{tcolorbox}   
        \end{column}
    % create the column with the second image, that also
    % occupies half of the slide
        \begin{column}{.4\textwidth}
        \begin{figure}
            \label{slope1}
            \centering
            \includegraphics[width=0.9\textwidth]{figures/section4.4/CalculusT1.png}
            \caption{$e^x$ with slope 1 at $(0,1)$}
        \end{figure}
        \end{column}
    \end{columns}

\end{frame}

%=====================================================================


\begin{frame}{Note on L'Hospital's Rule}

\begin{tcolorbox}[colframe=green!60!black, colback=white, title=\textbf{Note: One-Sided and Infinite Limits}]
L'Hospital's Rule is also valid for one-sided limits and for limits at infinity or negative infinity; that is, "$x \to a$" can be replaced by any of the symbols:
\[
x \to a^+, \quad x \to a^-, \quad x \to \infty, \quad \text{or} \quad x \to -\infty.
\]
\end{tcolorbox}

\end{frame}

%=====================================================================
\begin{frame}{Example: L'Hospital's Rule}

\textbf{Example.} Calculate $\displaystyle \lim_{x \to \infty} \frac{e^x}{x^2}$.

\vspace{0.5em}
\textbf{Solution.} We have $\displaystyle \lim_{x \to \infty} e^x = \infty$ and $\displaystyle \lim_{x \to \infty} x^2 = \infty$, so the limit is an indeterminate form of type $\frac{\infty}{\infty}$, and L'Hospital's Rule gives:
\[
\lim_{x \to \infty} \frac{e^x}{x^2} = \lim_{x \to \infty} \frac{ \frac{d}{dx}(e^x) }{ \frac{d}{dx}(x^2) } = \lim_{x \to \infty} \frac{e^x}{2x}
\]

Since $e^x \to \infty$ and $2x \to \infty$ as $x \to \infty$, the limit on the right side is also indeterminate.

\vspace{0.5em}
A second application of L'Hospital's Rule gives:

\[
\lim_{x \to \infty} \frac{e^x}{x^2} = \lim_{x \to \infty} \frac{e^x}{2x} = \lim_{x \to \infty} \frac{e^x}{2} = \infty
\]

\end{frame}
%=====================================================================
\begin{frame}{Example2: L'Hospital's Rule}

\textbf{Example.} Find $\displaystyle \lim_{x \to 0} \frac{\tan x - x}{x^3}$.

\end{frame}

%=====================================================================
\begin{frame}{Example2: L'Hospital's Rule}
\textbf{Solution.} Noting that both $\tan x - x \to 0$ and $x^3 \to 0$ as $x \to 0$, we use L'Hospital's Rule:
\[
\lim_{x \to 0} \frac{\tan x - x}{x^3}
= \lim_{x \to 0} \frac{\sec^2 x - 1}{3x^2}
\]

Since the limit on the right side is still of indeterminate form $\frac{0}{0}$, we apply L'Hospital's Rule again:
\[
\lim_{x \to 0} \frac{\sec^2 x - 1}{3x^2}
= \lim_{x \to 0} \frac{2\sec^2 x \tan x}{6x}
\]

\vspace{0.5em}
Because $\displaystyle \lim_{x \to 0} \sec^2 x = 1$, we simplify the calculation:
\[
\lim_{x \to 0} \frac{2\sec^2 x \tan x}{6x}
= \frac{1}{3} \lim_{x \to 0} \sec^2 x \cdot \lim_{x \to 0} \frac{\tan x}{x}
= \frac{1}{3} \cdot \lim_{x \to 0} \frac{\tan x}{x}
\]

\end{frame}
%=====================================================================
\begin{frame}{Example2: L'Hospital's Rule}

We can evaluate this last limit either by using L'Hospital's Rule a third time. Putting together all the steps, we get:
\[
\lim_{x \to 0} \frac{\tan x - x}{x^3}
= \lim_{x \to 0} \frac{\sec^2 x - 1}{3x^2}
= \lim_{x \to 0} \frac{2 \sec^2 x \tan x}{6x}
\]

\[
= \frac{1}{3} \lim_{x \to 0} \sec^2 x \cdot \lim_{x \to 0} \frac{\tan x}{x}
= \frac{1}{3} \cdot \lim_{x \to 0} \frac{\tan x}{x}
= \frac{1}{3} \cdot 1
= \frac{1}{3}
\]

\end{frame}

%=====================================================================
\begin{frame}{Indeterminate Form of Type $0 \cdot \infty$}

This kind of limit is called an \textbf{indeterminate form of type $0 \cdot \infty$}. We can deal with it by writing the product $fg$ as a quotient:
\[
fg = \frac{f}{1/g} \quad \text{or} \quad fg = \frac{g}{1/f}
\]

\begin{tcolorbox}[colframe=red!80!black, colback=white, title=\textbf{Strategy for Applying L'Hospital's Rule}]
This converts the given limit into an indeterminate form of type $\frac{0}{0}$ or $\frac{\infty}{\infty}$ so that we can use L'Hospital's Rule.
\end{tcolorbox}

\end{frame}

%=====================================================================
\begin{frame}{Example: Indeterminate Form $0 \cdot \infty$}

\textbf{Example.} Evaluate $\displaystyle \lim_{x \to 0^+} x \ln x$.

\vspace{0.5em}
\textbf{Solution.} The given limit is indeterminate because, as $x \to 0^+$, the first factor ($x$) approaches 0 while the second factor ($\ln x$) approaches $-\infty$.

We rewrite the product as a quotient:
\[
x \ln x = \frac{\ln x}{1/x}
\]

Then apply L'Hospital's Rule:
\[
\lim_{x \to 0^+} \frac{\ln x}{1/x}
= \lim_{x \to 0^+} \frac{1/x}{-1/x^2}
= \lim_{x \to 0^+} (-x)
= 0
\]
\end{frame}

%=====================================================================
\begin{frame}{Indeterminate Differences (Type $\infty - \infty$)}

\begin{tcolorbox}[colframe=blue!80!black, colback=white]
If $\displaystyle \lim_{x \to a} f(x) = \infty$ and $\displaystyle \lim_{x \to a} g(x) = \infty$, then the limit
\[
\lim_{x \to a} \left[ f(x) - g(x) \right]
\]
is called an \textbf{indeterminate form of type $\infty - \infty$}.
\end{tcolorbox}

\vspace{0.5em}
There is a contest between $f$ and $g$: will the result be $\infty$ (if $f$ wins), $-\infty$ (if $g$ wins), or a finite number? To resolve this form, we often try to convert the difference into a quotient — for example, by:
\begin{itemize}
  \item using a common denominator,
  \item rationalization, or
  \item factoring out a common factor
\end{itemize}

This way, we obtain an indeterminate form of type $\dfrac{0}{0}$ or $\dfrac{\infty}{\infty}$ to which L'Hospital's Rule can be applied.

\end{frame}

%=====================================================================
\begin{frame}{Example: Using a common denominator}

\textbf{Example.} Compute 
\[
\lim_{x \to 1^+} \left( \frac{1}{\ln x} - \frac{1}{x - 1} \right)
\]

\vspace{0.5em}
\textbf{Solution.} As $x \to 1^+$, both $\frac{1}{\ln x} \to \infty$ and $\frac{1}{x - 1} \to \infty$, so the limit is indeterminate of type $\infty - \infty$.

We begin by rewriting the expression with a common denominator:
\[
\lim_{x \to 1^+} \left( \frac{1}{\ln x} - \frac{1}{x - 1} \right)
= \lim_{x \to 1^+} \frac{x - 1 - \ln x}{(x - 1)\ln x}
\]

Both numerator and denominator approach 0, so we apply L'Hospital's Rule:
\[
\lim_{x \to 1^+} \frac{x - 1 - \ln x}{(x - 1)\ln x}
= \lim_{x \to 1^+} \frac{1 - \frac{1}{x}}{(x - 1) \cdot \frac{1}{x} + \ln x}
= \lim_{x \to 1^+} \frac{x - 1}{x \left( 1 + x \ln x \right)}
\]
\[
=\lim_{x \to 1^+} \frac{x - 1}{x \left( 1 + x \ln x \right)}
= \lim_{x \to 1^+} \frac{1}{1 + x \cdot \frac{1}{x} + \ln x}
= \lim_{x \to 1^+} \frac{1}{2 + \ln x}
= \frac{1}{2}
\]
\end{frame}

%=====================================================================
\begin{frame}{Example: factoring out a common factor}

\textbf{Example.} Calculate 
\[
\lim_{x \to \infty} (e^x - x)
\]

\vspace{0.5em}
\textbf{Solution.} This is an indeterminate difference because both $e^x$ and $x$ approach infinity.

We expect the limit to be $\infty$ since $e^x$ increases much faster than $x$. To confirm this, we factor out $x$:
\[
e^x - x = x \left( \frac{e^x}{x} - 1 \right)
\]

We observe that $\displaystyle \frac{e^x}{x} \to \infty$ as $x \to \infty$ by L'Hospital's Rule. Therefore, we now have a product of two large quantities:

\[
\lim_{x \to \infty} (e^x - x)
= \lim_{x \to \infty} \left[ x \left( \frac{e^x}{x} - 1 \right) \right]
= \infty
\]

\end{frame}

%=====================================================================
\begin{frame}{Indeterminate Powers (Types $0^0$, $\infty^0$, $1^\infty$)}

\begin{tcolorbox}[colframe=blue!80!black, colback=white, title=\textbf{Indeterminate Powers}]
Several indeterminate forms arise from the limit
\[
\lim_{x \to a} \left[ f(x) \right]^{g(x)}
\]
\begin{enumerate}
  \item $\displaystyle \lim_{x \to a} f(x) = 0$ and $\displaystyle \lim_{x \to a} g(x) = 0$ \hfill (type $0^0$)
  \item $\displaystyle \lim_{x \to a} f(x) = \infty$ and $\displaystyle \lim_{x \to a} g(x) = 0$ \hfill (type $\infty^0$)
  \item $\displaystyle \lim_{x \to a} f(x) = 1$ and $\displaystyle \lim_{x \to a} g(x) = \pm \infty$ \hfill (type $1^\infty$)
\end{enumerate}
\end{tcolorbox}

\end{frame}

%=====================================================================
\begin{frame}{Indeterminate Powers (Types $0^0$, $\infty^0$, $1^\infty$)}

Each of these cases can be treated by taking the natural logarithm. Let 
\[
y = \left[ f(x) \right]^{g(x)} \quad \Rightarrow \quad \ln y = g(x) \ln f(x)
\]

Alternatively, we can express the function as an exponential:
\[
\left[ f(x) \right]^{g(x)} = e^{g(x) \ln f(x)}
\]

This transforms the limit into a product $g(x) \ln f(x)$, which is of the indeterminate form $0 \cdot \infty$ and can be evaluated accordingly.
\end{frame}
%=====================================================================
\begin{frame}{Example: Indeterminate Form $1^\infty$}

\textbf{Example.} Calculate
\[
\lim_{x \to 0^+} \left(1 + \sin 4x\right)^{\cot x}
\]

\vspace{0.5em}
\textbf{Solution.} As $x \to 0^+$, we have $\sin 4x \to 0$ so $1 + \sin 4x \to 1$, and $\cot x \to \infty$.

Thus, the expression is of the indeterminate form $1^\infty$.

Let
\[
y = \left(1 + \sin 4x\right)^{\cot x}
\quad \Rightarrow \quad
\ln y = \cot x \cdot \ln(1 + \sin 4x)
= \frac{\ln(1 + \sin 4x)}{\tan x}
\]

Now apply L'Hospital's Rule:
\[
\lim_{x \to 0^+} \ln y
= \lim_{x \to 0^+} \frac{\ln(1 + \sin 4x)}{\tan x}
= \lim_{x \to 0^+} \frac{4 \cos 4x}{(1 + \sin 4x)\sec^2 x}
= 4
\]

So we have $\displaystyle \lim_{x \to 0^+} \ln y = 4$. Thus,
\[
\lim_{x \to 0^+} y = \lim_{x \to 0^+} e^{\ln y} = e^4
\]

\end{frame}

%=====================================================================

\end{document}