\documentclass{beamer}
\usepackage{graphicx} % Required for inserting images
\usepackage{amsmath}
\usepackage[most]{tcolorbox}
\usepackage{lmodern}
\usepackage{mathabx}

\usetheme{Madrid} % 可選其他主題:e.g., Warsaw, Berkeley, etc.
\usecolortheme{default}
\setbeamertemplate{caption}[numbered]% Number float-like environments
% customize the caption
\setbeamerfont{caption}{size=\footnotesize}
% \setbeamercolor{caption}{fg=blue}
% \setbeamercolor{caption name}{fg=red}

\title{Applications of Differentiation}
\subtitle{section4.9 : Antiderivatives}
\author{Hsu Chun-Wei}
\date{June 2025}

\begin{document}

\maketitle

%=====================================================================
\begin{frame}{Antiderivative}
\begin{tcolorbox}[colframe=red!80!black, colback=white, title=\textbf{Definition}]
A function $F$ is called an \textbf{antiderivative} of $f$ on an interval $I$ if
\[
F'(x) = f(x) \quad \text{for all } x \text{ in } I.
\]
\end{tcolorbox}

If two functions have identical derivatives on an interval, then they must differ by a constant. Therefore, if $F$ and $G$ are both antiderivatives of $f$ on $I$, then they must differ by a constant:
\[
G(x) = F(x) + C
\]
where $C$ is a constant. 
\end{frame}

%=====================================================================
\begin{frame}{Antiderivative}

\begin{columns}[c]
    \begin{column}{0.6 \textwidth}
        This leads to the following result:
        \begin{tcolorbox}[colframe=red!80!black, colback=white, title=\textbf{Theorem}]
        If $F$ is an antiderivative of $f$ on an interval $I$, then the most general antiderivative of $f$ on $I$ is
        \[
        F(x) + C
        \]
        where $C$ is an arbitrary constant.
        \end{tcolorbox}
    \end{column}
    \begin{column}{0.4\textwidth}
        \begin{figure}
            \centering
            \includegraphics[width=0.9\linewidth]{figures/section4.9/CalculusT1.png}
            \caption{Members of the family of antiderivatives of $f(x) =x^2$}
        \end{figure}
    \end{column}
\end{columns}

\end{frame}
%=====================================================================
\begin{frame}{Antidifferentiation Formulas}


\begin{center}

\resizebox{\textwidth}{!}{
\begin{tabular}{|c|c||c|c|}
\hline
\textbf{Function} & \textbf{Particular Antiderivative} & \textbf{Function} & \textbf{Particular Antiderivative} \\
\hline
$cf(x)$ & $cF(x)$ & $\sin x$ & $-\cos x$ \\
$f(x) + g(x)$ & $F(x) + G(x)$ & $\sec^2 x$ & $\tan x$ \\
$x^n \ (n \ne -1)$ & $\dfrac{x^{n+1}}{n+1}$ & $\sec x \tan x$ & $\sec x$ \\
$\dfrac{1}{x}$ & $\ln |x|$ & $\dfrac{1}{\sqrt{1 - x^2}}$ & $\sin^{-1}x$ \\
$e^x$ & $e^x$ & $\dfrac{1}{1 + x^2}$ & $\tan^{-1}x$ \\
$b^x$ & $\dfrac{b^x}{\ln b}$ & $\cosh x$ & $\sinh x$ \\
$\cos x$ & $\sin x$ & $\sinh x$ & $\cosh x$ \\
\hline
\end{tabular}
}

\end{center}

Try to check for these popular antidifferentiation Formulas
\end{frame}

%=====================================================================
\begin{frame}{Example: Solve $f'(x) = e^x + \frac{20}{1 + x^2}$ with $f(0) = -2$}
\textbf{Example:} Solve $f'(x) = e^x + \frac{20}{1 + x^2}$ with $f(0) = -2$

\textbf{Solution:}
\[
f(x) = \int f'(x)\,dx = \int e^x\,dx + \int \frac{20}{1 + x^2}\,dx = e^x + 20\tan^{-1}x + C
\]

Using Initial Condition $f(0) = -2$ to solve for $C$:
\[
f(0) = e^0 + 20\tan^{-1}0 + C = 1 + 0 + C = -2 \Rightarrow C = -3
\]

\begin{tcolorbox}[colframe=red!80!black, colback=white, title=\textbf{Final Answer}]
\[
f(x) = e^x + 20\tan^{-1}x - 3
\]
\end{tcolorbox}

\end{frame}

%=====================================================================
%=====================================================================
%=====================================================================
%=====================================================================
%=====================================================================
%=====================================================================
%=====================================================================
%=====================================================================

\end{document}