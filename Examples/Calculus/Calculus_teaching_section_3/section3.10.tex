\documentclass{beamer}
\usepackage{graphicx} % Required for inserting images
\usepackage{amsmath}
\usepackage[most]{tcolorbox}
\usepackage{lmodern}
\usepackage{mathabx}

\usetheme{Madrid} % 可選其他主題:e.g., Warsaw, Berkeley, etc.
\usecolortheme{default}
\setbeamertemplate{caption}[numbered]% Number float-like environments
% customize the caption
\setbeamerfont{caption}{size=\footnotesize}
% \setbeamercolor{caption}{fg=blue}
% \setbeamercolor{caption name}{fg=red}

\title{Differentiation Rules}
\subtitle{section3.10: Linear Approximations and Differentials}
\author{Hsu Chun-Wei}
\date{June 2025}

\begin{document}

\maketitle

%=====================================================================
\begin{frame}{Linearization and Approximation}
It might be easy to calculate a value $f(a)$ of a function, but difficult (or even impossible) to compute nearby values of $f$. So we settle for the easily computed values of the linear function $L$ whose graph is the tangent line of $f$ at $(a, f(a))$.

\vspace{0.5em}
In other words, we use the tangent line at $(a, f(a))$ as an approximation to the curve $y = f(x)$ when $x$ is near $a$. An equation of this tangent line is:
\[
y = f(a) + f'(a)(x - a)
\]

The linear function whose graph is this tangent line is:
\begin{tcolorbox}[colframe=red!80!black, colback=white, title=\textbf{Linearization of $f$ at $a$}]
\[
L(x) = f(a) + f'(a)(x - a)
\]
\end{tcolorbox}

\end{frame}
%=====================================================================
\begin{frame}{Linearization and Approximation}
\vspace{0.5em}
The approximation $f(x) \approx L(x)$, called the linear (or tangent line) approximation, is:
\begin{tcolorbox}[colframe=red!80!black, colback=white, title=\textbf{Linear Approximation}]
\[
f(x) \approx f(a) + f'(a)(x - a)
\]
\end{tcolorbox}
\begin{figure}
    \centering
    \includegraphics[width=0.35\linewidth]{figures/section3.10/CalculusT1.png}
    \caption{Linearization and Approximation}
    \label{Linearization and Approximation}
\end{figure}

\end{frame}
%=====================================================================
\begin{frame}{Differentials}

The ideas behind linear approximations are sometimes formulated in the terminology and notation of \textbf{differentials}. If $y = f(x)$, where $f$ is a differentiable function, then the \textbf{differential} $dx$ is an independent variable; that is, $dx$ can be given the value of any real number. 

The \textbf{differential} $dy$ is then defined in terms of $dx$ by the equation:

\begin{tcolorbox}[colframe=red!80!black, colback=white, title=\textbf{Definition of Differential}]
\[
dy = f'(x)\, dx
\]
\end{tcolorbox}

\vspace{0.5em}
So $dy$ is a dependent variable; it depends on the values of $x$ and $dx$. If $dx$ is given a specific value and $x$ is taken to be some specific number in the domain of $f$, then the numerical value of $dy$ is determined.

\end{frame}

%=====================================================================
\begin{frame}{Geometric Meaning of Differentials}

The geometric meaning of differentials is illustrated using two points: 
$P(x, f(x))$ and $Q(x + \Delta x, f(x + \Delta x))$ on the curve $y = f(x)$, 
and let $dx = \Delta x$. The corresponding change in $y$ is:
\[
\Delta y = f(x + \Delta x) - f(x)
\]

The slope of the tangent line $PR$ is the derivative $f'(x)$. 
Thus, the directed vertical distance from $S$ to $R$ is:
\[
f'(x) \cdot dx = dy
\]

$dy$ represents the change in the \textbf{tangent line (linearization)}.  
$\Delta y$ represents the actual change in $f(x)$ along the curve.

\end{frame}

%=====================================================================
\begin{frame}{Geometric Meaning of Differentials}
    \begin{figure}
        \centering
        \includegraphics[width=0.8\textwidth]{figures/section3.10/CalculusT2.png} % ← Replace with actual path
        \caption{Geometric Meaning of Differentials}
        \label{Geometric Meaning of Differentials}
    \end{figure}
\end{frame}
%=====================================================================
\begin{frame}{Example: Comparing $\Delta y$ and $dy$}

\textcolor{blue}{\textbf{EXAMPLE}} \quad
Compare the values of $\Delta y$ and $dy$ if $f(x) = x^3 + x^2 - 2x + 1$ and $x$ changes from 2 to 2.05 and from 2 to 2.01.

\vspace{0.5em}
\textcolor{blue}{\textbf{SOLUTION}}

(a) When $x$ changes from 2 to 2.05:

\[
f(2) = 2^3 + 2^2 - 2(2) + 1 = 9
\]
\[
f(2.05) = (2.05)^3 + (2.05)^2 - 2(2.05) + 1 = 9.717625
\]
\[
\Delta y = f(2.05) - f(2) = 0.717625
\]

General form:
\[
dy = f'(x)\, dx = (3x^2 + 2x - 2) dx
\]
At $x = 2$, $dx = 0.05$:
\[
dy = [3(2)^2 + 2(2) - 2] \cdot 0.05 = 0.7
\]

\end{frame}

%=====================================================================
\begin{frame}{Example: Comparing $\Delta y$ and $dy$}
(b) When $x$ changes from 2 to 2.01:

\[
f(2.01) = (2.01)^3 + (2.01)^2 - 2(2.01) + 1 = 9.140701
\]
\[
\Delta y = f(2.01) - f(2) = 0.140701
\]

At $dx = 0.01$:
\[
dy = [3(2)^2 + 2(2) - 2] \cdot 0.01 = 0.14
\]

\vspace{0.5em}
\textcolor{blue}{\textbf{Observation:}} As $\Delta x$ becomes smaller, the approximation $\Delta y \approx dy$ improves. Also, $dy$ is easier to compute than $\Delta y$.
\end{frame}
%=====================================================================
\begin{frame}{}
    \begin{figure}
        \centering
        \includegraphics[width=0.8\linewidth]{figures/section3.10/CalculusT3.png}
        \caption{Comparing $\Delta y$ and $dy$}
        \label{Comparing $\Delta y$ and $dy$}
    \end{figure}
\end{frame}
%=====================================================================
%=====================================================================
%=====================================================================
%=====================================================================

\end{document}