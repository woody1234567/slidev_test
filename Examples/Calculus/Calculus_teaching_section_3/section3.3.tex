\documentclass{beamer}
\usepackage{graphicx} % Required for inserting images
\usepackage{amsmath}
\usepackage[most]{tcolorbox}
\usepackage{lmodern}
\usepackage{mathabx}

\usetheme{Madrid} % 可選其他主題:e.g., Warsaw, Berkeley, etc.
\usecolortheme{default}
\setbeamertemplate{caption}[numbered]% Number float-like environments
% customize the caption
\setbeamerfont{caption}{size=\footnotesize}
% \setbeamercolor{caption}{fg=blue}
% \setbeamercolor{caption name}{fg=red}

\title{Differentiation Rules}
\subtitle{section3.3: Derivatives of Trigonometric Functions}
\author{Hsu Chun-Wei}
\date{June 2025}

\begin{document}

\maketitle

%=====================================================================
\begin{frame}{Derivative of $sin x$}

Let’s try to confirm our guess that if \( f(x) = \sin x \), then \( f'(x) = \cos x \).  
From the definition of a derivative, we have

\[
f'(x) = \lim_{h \to 0} \frac{f(x+h) - f(x)}{h}
= \lim_{h \to 0} \frac{\sin(x+h) - \sin x}{h}
\]

\[
= \lim_{h \to 0} \frac{\sin x \cos h + \cos x \sin h - \sin x}{h}
\]

\[
= \lim_{h \to 0} \left[
\frac{\sin x \cos h - \sin x}{h}
+ \frac{\cos x \sin h}{h}
\right]
\]

\[
= \lim_{h \to 0} \left[
\sin x \left( \frac{\cos h - 1}{h} \right)
+ \cos x \left( \frac{\sin h}{h} \right)
\right]
\]

\[
= \lim_{h \to 0}\sin x \cdot \lim_{h \to 0} \frac{\cos h - 1}{h}
+ \lim_{h \to 0}\cos x \cdot \lim_{h \to 0} \frac{\sin h}{h}
\]

\end{frame}

%=====================================================================
\begin{frame}{Derivative of $sin x$}

Two of these four limits are easy to evaluate.  
Because we regard \( x \) as a constant when computing a limit as \( h \to 0 \), we have

\[
\lim_{h \to 0} \sin x = \sin x
\qquad \text{and} \qquad
\lim_{h \to 0} \cos x = \cos x
\]

Later in this section we will prove that

\[
\lim_{h \to 0} \frac{\sin h}{h} = 1
\qquad \text{and} \qquad
\lim_{h \to 0} \frac{\cos h - 1}{h} = 0
\]

So we have proved the formula for the derivative of the sine function:

\begin{tcolorbox}[colframe=red!80!black, colback=white,
  title=\textbf{Derivative of the Sine Function} ]
\[
\frac{d}{dx} (\sin x) = \cos x
\]
\end{tcolorbox}
\end{frame}
%=====================================================================
\begin{frame}{Derivative of $\sin x$}
\begin{figure}
    \centering
    \includegraphics[width=0.8\linewidth]{figures/section3.3/CalculusT1.png}
    \caption{The relationship between the derivative of sin x and cos x}
    \label{$sin' (x) = cos(x)$}
\end{figure}
\end{frame}
%=====================================================================
\begin{frame}{Derivative of $\cos x$}

\textbf{Problem 26.} Prove, using the definition of a derivative, that if  
\( f(x) = \cos x \), then \( f'(x) = -\sin x \).

\vspace{1em}

By the definition of the derivative:

\[
f'(x) = \lim_{h \to 0} \frac{f(x+h) - f(x)}{h}
= \lim_{h \to 0} \frac{\cos(x+h) - \cos x}{h}
\]

\[
= \lim_{h \to 0} \frac{\cos x \cos h - \sin x \sin h - \cos x}{h}
\]

\[
= \lim_{h \to 0} \left[
\frac{\cos x (\cos h - 1)}{h}
- \frac{\sin x \sin h}{h}
\right]
\]

\[
= \cos x \cdot \lim_{h \to 0} \frac{\cos h - 1}{h}
- \sin x \cdot \lim_{h \to 0} \frac{\sin h}{h}
\]

\[
= \cos x \cdot 0 - \sin x \cdot 1 = -\sin x
\]

\end{frame}
%=====================================================================
\begin{frame}{Limit Proof: $\displaystyle \lim_{\theta \to 0} \frac{\sin \theta }{\theta} = 1$}

\begin{tcolorbox}[colframe=red!80!black, colback=white,
  title=\textbf{Limit identity} ]
\[
\lim_{\theta \to 0} \frac{\sin \theta}{\theta} = 1
\]
\end{tcolorbox}

\vspace{1em}

\begin{columns}[c]
% create the column with the first image, that occupies
% half of the slide
    \begin{column}{.5\textwidth}
    \textbf{\color{red}PROOF} \quad Assume first that \( \theta \) lies between \( 0 \) and \( \pi/2 \).  
    Figure \ref{proof1} shows a sector of a circle with center \( O \), central angle \( \theta \), and radius 1.  
    \( BC \) is drawn perpendicular to \( OA \). By the definition of radian measure, we have arc \( AB = \theta \).  
    Also \( |BC| = |OB| \sin \theta = \sin \theta \). 
    \end{column}
% create the column with the second image, that also
% occupies half of the slide
    \begin{column}{.5\textwidth}
    \begin{figure}
        \centering
        \includegraphics[width=0.5\linewidth]{figures/section3.3/CalculusT2.png}
        \caption{proof for $\lim_{\theta \to 0} \frac{\sin \theta}{\theta} = 1$}
        \label{proof1}
    \end{figure}
    \end{column}
\end{columns}


\end{frame}

%=====================================================================
\begin{frame}{Limit Proof: $\displaystyle \lim_{\theta \to 0} \frac{\sin \theta }{\theta} = 1$}
From the diagram we see that
    \[
    |BC| < |AB| < \text{arc } AB
    \]   
Therefore, \( \sin \theta < \theta \), so

\[
\frac{\sin \theta}{\theta} < 1
\]

Let the tangent lines at \( A \) and \( B \) intersect at \( E \).

\[
\theta = \text{arc } AB < |AE| + |EB|
< |AE| + |ED|
= |AD| = |OA| \tan \theta = \tan \theta
\]

\end{frame}
%=====================================================================
\begin{frame}{Limit Proof: $\displaystyle \lim_{\theta \to 0} \frac{\sin \theta }{\theta} = 1$}

Therefore, we have

\[
\theta < \frac{\sin \theta}{\cos \theta}
\quad \text{so} \quad
\cos \theta < \frac{\sin \theta}{\theta} < 1
\]

We know that \( \lim_{\theta \to 0} 1 = 1 \) and \( \lim_{\theta \to 0} \cos \theta = 1 \), so by the Squeeze Theorem, we have

\[
\lim_{\theta \to 0^+} \frac{\sin \theta}{\theta} = 1
\quad \text{(for \( 0 < \theta < \frac{\pi}{2} \))}
\]

But the function \( \frac{\sin \theta}{\theta} \) is an even function, so its right and left limits must be equal.  
Hence, we have

\[
\lim_{\theta \to 0} \frac{\sin \theta}{\theta} = 1
\]

\end{frame}

%=====================================================================
\begin{frame}{Limit Proof: $\displaystyle \lim_{\theta \to 0} \frac{\cos \theta - 1}{\theta} = 0$}

\begin{tcolorbox}[colframe=red!80!black, colback=white,
  title=\textbf{Limit Identity}]
\[
\lim_{\theta \to 0} \frac{\cos \theta - 1}{\theta} = 0
\]
\end{tcolorbox}

\textcolor{red}{\textbf{PROOF:}} We multiply numerator and denominator by $\cos \theta + 1$ in order to put the function in a form in which we can use limits that we know.
\[
\begin{aligned}
\lim_{\theta \to 0} \frac{\cos \theta - 1}{\theta}
&= \lim_{\theta \to 0} \left( \frac{\cos \theta - 1}{\theta} \cdot \frac{\cos \theta + 1}{\cos \theta + 1} \right)= \lim_{\theta \to 0} \frac{\cos^2 \theta - 1}{\theta(\cos \theta + 1)} \\
&= \lim_{\theta \to 0} \frac{-\sin^2 \theta}{\theta(\cos \theta + 1)} = -\lim_{\theta \to 0} \left( \frac{\sin \theta}{\theta} \cdot \frac{\sin \theta}{\cos \theta + 1} \right) \\
&= -\lim_{\theta \to 0} \frac{\sin \theta}{\theta} \cdot \lim_{\theta \to 0} \frac{\sin \theta}{\cos \theta + 1} \\
&= -1 \cdot \left( \frac{0}{1 + 1} \right) = 0
\end{aligned}
\]

\end{frame}
%=====================================================================
\begin{frame}{Derivatives of Trigonometric Functions}

\begin{tcolorbox}[colframe=red!80!black, colback=white,
  title=\textbf{Derivatives of Trigonometric Functions} ]
\[
\begin{aligned}
\frac{d}{dx} (\sin x) &= \cos x
&\qquad \frac{d}{dx} (\csc x) &= -\csc x \cot x \\
\frac{d}{dx} (\cos x) &= -\sin x
&\qquad \frac{d}{dx} (\sec x) &= \sec x \tan x \\
\frac{d}{dx} (\tan x) &= \sec^2 x
&\qquad \frac{d}{dx} (\cot x) &= -\csc^2 x
\end{aligned}
\]
\end{tcolorbox}

Try to prove of all of the equations from the given properties of derivatives of trigonometric function.

\end{frame}
%=====================================================================
%=====================================================================
%=====================================================================
%=====================================================================

\end{document}