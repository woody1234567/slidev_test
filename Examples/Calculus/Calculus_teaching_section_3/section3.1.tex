\documentclass{beamer}
\usepackage{graphicx} % Required for inserting images
\usepackage{amsmath}
\usepackage[most]{tcolorbox}
\usepackage{lmodern}
\usepackage{mathabx}

\usetheme{Madrid} % 可選其他主題:e.g., Warsaw, Berkeley, etc.
\usecolortheme{default}
\setbeamertemplate{caption}[numbered]% Number float-like environments
% customize the caption
\setbeamerfont{caption}{size=\footnotesize}
% \setbeamercolor{caption}{fg=blue}
% \setbeamercolor{caption name}{fg=red}

\title{Differentiation Rules}
\subtitle{section3.1: Derivatives of Polynomials and Exponential Functions}
\author{Hsu Chun-Wei}
\date{June 2025}

\begin{document}

\maketitle

%=====================================================================
\begin{frame}{Derivative of a Constant Function}

\begin{tcolorbox}[colframe=red!80!black, colback=white,  title=Derivative of a Constant Function]
\[
\frac{d}{dx}(c) = 0
\]
\end{tcolorbox}

\vspace{1em}

\textbf{proof:}

\[
f'(x) = \lim_{h \to 0} \frac{f(x+h) - f(x)}{h}
= \lim_{h \to 0} \frac{c - c}{h}
= \lim_{h \to 0} 0
= 0
\]

\end{frame}
%=====================================================================
\begin{frame}{The Power Rule}

\begin{tcolorbox}[colframe=red!80!black, colback=white,
  title=\textbf{The Power Rule}]
  
If $n$ is a positive integer, then
\[
\frac{d}{dx}(x^n) = nx^{n-1}
\]
\end{tcolorbox}


\end{frame}
%=====================================================================
\begin{frame}{Proof of the Power Rule}

\[
f'(x) = \lim_{h \to 0} \frac{f(x+h) - f(x)}{h} 
= \lim_{h \to 0} \frac{(x+h)^n - x^n}{h}
\]

\vspace{0.5em}

Here we need to expand \( (x+h)^n \) and we use the Binomial Theorem to do so:

\[
f'(x) = \lim_{h \to 0} \frac{ 
\left[
x^n + nx^{n-1}h + \frac{n(n-1)}{2}x^{n-2}h^2 + \cdots + nxh^{n-1} + h^n
\right] - x^n }{h}
\]

\[
= \lim_{h \to 0} \frac{
nx^{n-1}h + \frac{n(n-1)}{2}x^{n-2}h^2 + \cdots + nxh^{n-1} + h^n }{h}
\]

\[
= \lim_{h \to 0} \left[
nx^{n-1} + \frac{n(n-1)}{2}x^{n-2}h + \cdots + nxh^{n-2} + h^{n-1}
\right]
\]

\[
= nx^{n-1}
\]

\end{frame}
%=====================================================================
\begin{frame}{The Constant Multiple Rule}

\begin{tcolorbox}[colframe=red!80!black, colback=white,
  title=\textbf{The Constant Multiple Rule} ]
If $c$ is a constant and $f$ is a differentiable function, then
\[
\frac{d}{dx}[cf(x)] = c \frac{d}{dx} f(x)
\]
\end{tcolorbox}

\vspace{1em}
\textbf{\color{red}PROOF} \quad Let \( g(x) = cf(x) \). Then

\[
g'(x) = \lim_{h \to 0} \frac{g(x+h) - g(x)}{h}
= \lim_{h \to 0} \frac{cf(x+h) - cf(x)}{h}
\]

\[
= \lim_{h \to 0} c \left[ \frac{f(x+h) - f(x)}{h} \right]
= c \lim_{h \to 0} \frac{f(x+h) - f(x)}{h} = c f'(x)
\]

\end{frame}
%=====================================================================
\begin{frame}{The Sum and Difference Rules}

\begin{tcolorbox}[colframe=red!80!black, colback=white,
  title=\textbf{The Sum and Difference Rules} ]
If $f$ and $g$ are both differentiable, then
\[
\frac{d}{dx}[f(x) + g(x)] = \frac{d}{dx}f(x) + \frac{d}{dx}g(x)
\]
\[
\frac{d}{dx}[f(x) - g(x)] = \frac{d}{dx}f(x) - \frac{d}{dx}g(x)
\]
\end{tcolorbox}

\vspace{1em}
\textbf{\color{red}PROOF} \quad To prove the Sum Rule, we let \( F(x) = f(x) + g(x) \). Then

\[
F'(x) = \lim_{h \to 0} \frac{F(x+h) - F(x)}{h}
= \lim_{h \to 0} \frac{[f(x+h) + g(x+h)] - [f(x) + g(x)]}{h}
\]

\[
= \lim_{h \to 0} \frac{f(x+h) - f(x)}{h}
+ \lim_{h \to 0} \frac{g(x+h) - g(x)}{h}
= f'(x) + g'(x)
\]

\end{frame}

%=====================================================================
\begin{frame}{Exponential Functions}

Let’s try to compute the derivative of the exponential function \( f(x) = b^x \) using the definition of a derivative:

\[
f'(x) = \lim_{h \to 0} \frac{f(x+h) - f(x)}{h}
= \lim_{h \to 0} \frac{b^{x+h} - b^x}{h}
\]

\[
= \lim_{h \to 0} \frac{b^x b^h - b^x}{h}
= \lim_{h \to 0} \frac{b^x (b^h - 1)}{h}
\]

The factor \( b^x \) doesn’t depend on \( h \), so we can take it in front of the limit:

\[
f'(x) = b^x \lim_{h \to 0} \frac{b^h - 1}{h}
\]
\end{frame}

%=====================================================================
\begin{frame}{Exponential Functions}

Notice that the limit is the value of the derivative of \( f \) at 0, that is,

\[
\lim_{h \to 0} \frac{b^h - 1}{h} = f'(0)
\]

Therefore we have shown that if the exponential function \( f(x) = b^x \) is differentiable at 0, then it is differentiable everywhere and

\[
f'(x) = f'(0) b^x
\]

This equation says that \textbf{the rate of change of any exponential function is proportional to the function itself}.

\end{frame}
%=====================================================================
\begin{frame}{Definition of the Number $e$}

\begin{tcolorbox}[colframe=red!80!black, colback=white,
  title=\textbf{Definition of the Number $e$} ]
$e$ is the number such that
\[
\lim_{h \to 0} \frac{e^h - 1}{h} = 1
\]
\end{tcolorbox}

\vspace{1em}

Geometrically, this means that of all possible exponential functions \( y = b^x \),  
the function \( f(x) = e^x \) is the one whose tangent line at \( (0,1) \) has a slope \( f'(0) \) that is exactly 1.  

\vspace{1em}
\textit{(See Figures \ref{between 2 and 3}. and \ref{slope1}.)}

\end{frame}
%=====================================================================
\begin{frame}{Definition of the Number $e$}
    \begin{columns}[c]
% create the column with the first image, that occupies
% half of the slide
        \begin{column}{.5\textwidth}
        \begin{figure}
            \label{between 2 and 3}
            \centering
            \includegraphics[width=0.8\textwidth]{figures/section3.1/CalculusT1.png}
            \caption{The graph of $b^x$ where $2 < b < 3$}
        \end{figure}      
        \end{column}
    % create the column with the second image, that also
    % occupies half of the slide
        \begin{column}{.5\textwidth}
        \begin{figure}
            \label{slope1}
            \centering
            \includegraphics[width=0.9\textwidth]{figures/section3.1/CalculusT2.png}
            \caption{$e^x$ with slope 1 at $(0,1)$}
        \end{figure}
        \end{column}
    \end{columns}
\end{frame}
%=====================================================================
\begin{frame}{Definition of the Number $e$}
If we put \( b = e \) and therefore \( f'(0) = 1 \) in Equation 4, it becomes the following important differentiation formula:

\[
f'(x) = f'(0) \cdot b^x = 1 \cdot e^x = e^x
\]

\begin{tcolorbox}[colframe=red!80!black, colback=white,
  title=\textbf{Derivative of the Natural Exponential Function} ]
\[
\frac{d}{dx}(e^x) = e^x
\]
\end{tcolorbox}

\end{frame}
%=====================================================================
\begin{frame}{Derivatives of General Exponential Functions}

We can use the Chain Rule to differentiate an exponential function with any base \( b > 0 \).  
Recall from Equation 1.5.10 that we can write

\[
b^x = e^{(\ln b)x}
\]

and then the Chain Rule gives

\[
\frac{d}{dx} (b^x)
= \frac{d}{dx} \left( e^{(\ln b)x} \right)
= e^{(\ln b)x} \cdot \frac{d}{dx}[(\ln b)x]
\]

\[
= e^{(\ln b)x} \cdot (\ln b)
= b^x \ln b
\]

because \( \ln b \) is a constant. So we have the formula:

\begin{tcolorbox}[colframe=red!80!black, colback=white,
  title=\textbf{Derivative of General Exponential Function} ]
\[
\frac{d}{dx} (b^x) = b^x \ln b
\]
\end{tcolorbox}

\end{frame}
%=====================================================================

\end{document}
