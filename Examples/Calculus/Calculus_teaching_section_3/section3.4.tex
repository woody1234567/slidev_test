\documentclass{beamer}
\usepackage{graphicx} % Required for inserting images
\usepackage{amsmath}
\usepackage[most]{tcolorbox}
\usepackage{lmodern}
\usepackage{mathabx}

\usetheme{Madrid} % 可選其他主題:e.g., Warsaw, Berkeley, etc.
\usecolortheme{default}
\setbeamertemplate{caption}[numbered]% Number float-like environments
% customize the caption
\setbeamerfont{caption}{size=\footnotesize}
% \setbeamercolor{caption}{fg=blue}
% \setbeamercolor{caption name}{fg=red}

\title{Differentiation Rules}
\subtitle{section3.4: The Chain Rule}
\author{Hsu Chun-Wei}
\date{June 2025}

\begin{document}

\maketitle

%=====================================================================
\begin{frame}{The Chain Rule}

\begin{tcolorbox}[colframe=red!80!black, colback=white,
  title=\textbf{The Chain Rule}]

\textbf{\textcolor{red}{The Chain Rule}} 
If $g$ is differentiable at $x$ and $f$ is differentiable at $g(x)$, then the composite function 
$F = f \circ g$ defined by $F(x) = f(g(x))$ is differentiable at $x$ and $F'$ is given by the product

\vspace{0.5em}
\textcolor{red}{\fbox{\textbf{1}}} \qquad
$\displaystyle F'(x) = f'(g(x)) \cdot g'(x)$

\vspace{1em}
In Leibniz notation, if $y = f(u)$ and $u = g(x)$ are both differentiable functions, then

\vspace{0.5em}
\textcolor{red}{\fbox{\textbf{2}}} \qquad
$\displaystyle \frac{dy}{dx} = \frac{dy}{du} \cdot \frac{du}{dx}$

\end{tcolorbox}

\end{frame}
%=====================================================================
\begin{frame}{Key background to prove the Chain Rule}

Recall that if $y = f(x)$ and $x$ changes from $a$ to $a + \Delta x$, we define the increment of $y$ as
\[
\Delta y = f(a + \Delta x) - f(a)
\]

According to the definition of a derivative, we have
\[
\lim_{\Delta x \to 0} \frac{\Delta y}{\Delta x} = f'(a)
\]

So if we denote by $\varepsilon$ the difference between $\frac{\Delta y}{\Delta x}$ and $f'(a)$, we obtain
\[
\lim_{\Delta x \to 0} \varepsilon 
= \lim_{\Delta x \to 0} \left( \frac{\Delta y}{\Delta x} - f'(a) \right) 
= f'(a) - f'(a) = 0
\]

\[
\varepsilon = \frac{\Delta y}{\Delta x} - f'(a)
\quad \Rightarrow \quad
\Delta y = f'(a) \Delta x + \varepsilon \Delta x
\]

If we define $\varepsilon$ to be $0$ when $\Delta x = 0$, then $\varepsilon$ becomes a continuous function of $\Delta x$. Thus, for a differentiable function $f$, we can write

\[
\Delta y = f'(a)\, \Delta x + \varepsilon\, \Delta x
\quad \text{where} \quad \varepsilon \to 0 \text{ as } \Delta x \to 0
\]

\end{frame}

%=====================================================================
\begin{frame}{Proof of the Chain Rule}

Suppose $u = g(x)$ is differentiable at $a$ and $y = f(u)$ is differentiable at $b = g(a)$. If $\Delta x$ is an increment in $x$ and $\Delta u$, $\Delta y$ are the corresponding increments in $u$ and $y$, then:

\begin{equation}
\label{proof of chain rule 1}
    \Delta u = g'(a)\, \Delta x + \varepsilon_1\, \Delta x 
    = \left[ g'(a) + \varepsilon_1 \right] \Delta x
\end{equation}

where $\varepsilon_1 \to 0$ as $\Delta x \to 0$.

Similarly,

\begin{equation}
\label{proof of chain rule 2}
\Delta y = f'(b)\, \Delta u + \varepsilon_2\, \Delta u 
= \left[ f'(b) + \varepsilon_2 \right] \Delta u
\end{equation}

where $\varepsilon_2 \to 0$ as $\Delta u \to 0$.

If we substitute the expression for $\Delta u$ from Equation \ref{proof of chain rule 1} into Equation \ref{proof of chain rule 2}, we get
\[
\Delta y = \left[ f'(b) + \varepsilon_2 \right] \left[ g'(a) + \varepsilon_1 \right] \Delta x
\]

\end{frame}
%=====================================================================
\begin{frame}{proof of the Chain rule}
    so
\[
\frac{\Delta y}{\Delta x} = \left[ f'(b) + \varepsilon_2 \right] \left[ g'(a) + \varepsilon_1 \right]
\]

As $\Delta x \to 0$, Equation 7 shows that $\Delta u \to 0$. Taking the limit:
\[
\frac{dy}{dx} = \lim_{\Delta x \to 0} \frac{\Delta y}{\Delta x}
= \lim_{\Delta x \to 0} \left[ f'(b) + \varepsilon_2 \right] \left[ g'(a) + \varepsilon_1 \right]
\]

\[
= f'(b)\, g'(a) = f'(g(a))\, g'(a)
\]

This proves the Chain Rule.
\end{frame}
%=====================================================================
%=====================================================================
%=====================================================================
%=====================================================================
%=====================================================================
%=====================================================================
%=====================================================================
%=====================================================================

\end{document}