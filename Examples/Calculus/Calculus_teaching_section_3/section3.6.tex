\documentclass{beamer}
\usepackage{graphicx} % Required for inserting images
\usepackage{amsmath}
\usepackage[most]{tcolorbox}
\usepackage{lmodern}
\usepackage{mathabx}

\usetheme{Madrid} % 可選其他主題:e.g., Warsaw, Berkeley, etc.
\usecolortheme{default}
\setbeamertemplate{caption}[numbered]% Number float-like environments
% customize the caption
\setbeamerfont{caption}{size=\footnotesize}
% \setbeamercolor{caption}{fg=blue}
% \setbeamercolor{caption name}{fg=red}

\title{Differentiation Rules}
\subtitle{section3.6: Derivatives of Logarithmic and Inverse Trigonometric Functions}
\author{Hsu Chun-Wei}
\date{June 2025}

\begin{document}

\maketitle

%=====================================================================
\begin{frame}{Derivative of $\log_b x$}

\begin{tcolorbox}[colframe=red!80!black, colback=white,
  title=\textbf{Logarithmic Derivative Formula}]
  
\begin{equation}
\label{Derivative of log_b(x)}
\frac{d}{dx} (\log_b x) = \frac{1}{x \ln b}
\end{equation}

\end{tcolorbox}

\textcolor{red}{\textbf{PROOF}} \quad Let $y = \log_b x$. Then
\[
b^y = x
\]

Differentiating this equation implicitly with respect to $x$, we get:
\[
(b^y \ln b) \frac{dy}{dx} = 1
\]

and so
\[
\frac{dy}{dx} = \frac{1}{b^y \ln b} = \frac{1}{x \ln b}
\]

\end{frame}
%=====================================================================
\begin{frame}{}

If we put $b = e$ in Equation \ref{Derivative of log_b(x)}, then the factor $\ln b$ on the right becomes $\ln e = 1$, and we get the formula for the derivative of the natural logarithmic function $\log_e x = \ln x$:

\begin{tcolorbox}[colframe=red!80!black, colback=white,
  title=\textbf{The derivative of the natural logarithmic function}]
  
\[
\frac{d}{dx} (\ln x) = \frac{1}{x}
\]

\end{tcolorbox}

\end{frame}
%=====================================================================
\begin{frame}{Example: Derivative of $\ln|x|$}

\textcolor{blue}{\textbf{EXAMPLE}} \quad Find $f'(x)$ if $f(x) = \ln |x|$.

\vspace{0.5em}
\textcolor{blue}{\textbf{SOLUTION}} \quad Since
\[
f(x) = 
\begin{cases}
\ln x & \text{if } x > 0 \\
\ln(-x) & \text{if } x < 0
\end{cases}
\]

it follows that
\[
f'(x) = 
\begin{cases}
\frac{1}{x} & \text{if } x > 0 \\
\frac{1}{-x}(-1) = \frac{1}{x} & \text{if } x < 0
\end{cases}
\]

\begin{tcolorbox}[colframe=red!80!black, colback=white, title=\textbf{Derivative of $\ln|x|$}]
\[
\frac{d}{dx} \ln |x| = \frac{1}{x}
\]
\end{tcolorbox}

\end{frame}

%=====================================================================
\begin{frame}{Logarithmic Differentiation}

The calculation of derivatives of complicated functions involving products, quotients, or powers can often be simplified by taking logarithms. The method used is called \textbf{logarithmic differentiation}.

\begin{tcolorbox}[colframe=red!80!black, colback=white,
  title=\textbf{Steps in Logarithmic Differentiation}]

\begin{enumerate}
  \item Take natural logarithms of both sides of an equation $y = f(x)$ and use the Laws of Logarithms to expand the expression.
  \item Differentiate implicitly with respect to $x$.
  \item Solve the resulting equation for $y'$ and replace $y$ by $f(x)$.
\end{enumerate}

\end{tcolorbox}

\end{frame}

%=====================================================================
\begin{frame}{Logarithmic Differentiation Example}
\textcolor{blue}{\textbf{EXAMPLE}} \quad Differentiate 
\[
y = \frac{x^{3/4} \sqrt{x^2 + 1}}{(3x + 2)^5}
\]

\textcolor{blue}{\textbf{SOLUTION}} \quad
We take logarithms of both sides and use the Laws of Logarithms to simplify:
\[
\ln y = \frac{3}{4} \ln x + \frac{1}{2} \ln(x^2 + 1) - 5 \ln(3x + 2)
\]

Differentiating implicitly with respect to $x$:
\[
\frac{1}{y} \frac{dy}{dx} = \frac{3}{4} \cdot \frac{1}{x}
+ \frac{1}{2} \cdot \frac{2x}{x^2 + 1}
- 5 \cdot \frac{3}{3x + 2}
\]

Solving for $\frac{dy}{dx}$:
\[
\frac{dy}{dx} = y \left( \frac{3}{4x} + \frac{x}{x^2 + 1} - \frac{15}{3x + 2} \right) = \frac{x^{3/4} \sqrt{x^2 + 1}}{(3x + 2)^5}
\left( \frac{3}{4x} + \frac{x}{x^2 + 1} - \frac{15}{3x + 2} \right)
\]

\end{frame}
%=====================================================================
\begin{frame}{The Number $e$ as a Limit}

We have shown that if $f(x) = \ln x$, then $f'(x) = \frac{1}{x}$. Thus $f'(1) = 1$. We now use this fact to express the number $e$ as a limit.

\vspace{0.5em}
From the definition of a derivative as a limit, we have:
\[
f'(1) = \lim_{h \to 0} \frac{f(1 + h) - f(1)}{h}
= \lim_{x \to 0} \frac{f(1 + x) - f(1)}{x}
\]

\[
= \lim_{x \to 0} \frac{\ln(1 + x) - \ln 1}{x}
= \lim_{x \to 0} \frac{1}{x} \ln(1 + x)= \lim_{x \to 0} \ln(1 + x)^{1/x}
\]

Because $f'(1) = 1$, we have
\[
\lim_{x \to 0} \ln(1 + x)^{1/x} = 1
\]

Then:

\[
e = e^1 = e^{\lim_{x \to 0} \ln(1 + x)^{1/x}}
= \lim_{x \to 0} e^{\ln(1 + x)^{1/x}}
= \lim_{x \to 0} (1 + x)^{1/x}
\]

\end{frame}

%=====================================================================
\begin{frame}{Definition of $e$ as a Limit}

\begin{tcolorbox}[colframe=red!80!black, colback=white, title=\textbf{Definition of $e$}]
\[
e = \lim_{x \to 0} (1 + x)^{1/x}
\]
\end{tcolorbox}
If we put $n = 1/x$, then $n \to \infty$ as $x \to 0^+$, and so an alternative expression for $e$ is:

\begin{tcolorbox}[colframe=red!80!black, colback=white, title=\textbf{Classic Limit of $e$}]
\[
e = \lim_{n \to \infty} \left(1 + \frac{1}{n} \right)^n
\]
\end{tcolorbox}

\end{frame}
%=====================================================================
\begin{frame}{Definition of $e$ as a Limit}
    \begin{columns}[c]
% create the column with the first image, that occupies
% half of the slide
        \begin{column}{.5\textwidth}
        \begin{figure}
            \label{between 2 and 3}
            \centering
            \includegraphics[width=0.8\textwidth]{figures/section3.6/CalculusT1.png}
            \caption{The graph of $y=(1+x)^{1/x}$}
        \end{figure}      
        \end{column}
    % create the column with the second image, that also
    % occupies half of the slide
        \begin{column}{.5\textwidth}
        \begin{figure}
            \label{slope1}
            \centering
            \includegraphics[width=0.9\textwidth]{figures/section3.6/CalculusT2.png}
            \caption{Numerical results of $y=(1+x)^{1/x}$}
        \end{figure}
        \end{column}
    \end{columns}
\end{frame}
%=====================================================================
\begin{frame}{Derivative of the Inverse Sine Function}

Recall the definition of the arcsine function:
\[
y = \sin^{-1} x \quad \text{means} \quad \sin y = x \quad \text{and} \quad -\frac{\pi}{2} \le y \le \frac{\pi}{2}
\]

Differentiating $\sin y = x$ implicitly with respect to $x$, we obtain:
\[
\cos y \, \frac{dy}{dx} = 1 \quad \text{or} \quad \frac{dy}{dx} = \frac{1}{\cos y}
\]

Now $\cos y \ge 0$ because $-\frac{\pi}{2} \le y \le \frac{\pi}{2}$, so
\[
\cos y = \sqrt{1 - \sin^2 y} = \sqrt{1 - x^2} \quad \text{\small (since } \cos^2 y + \sin^2 y = 1\text{)}
\]

\begin{tcolorbox}[colframe=red!80!black, colback=white, title=\textbf{Derivative of $\sin^{-1}x$}]
\[
\frac{dy}{dx}=\frac{d}{dx} \left( \sin^{-1} x \right) = \frac{1}{\sqrt{1 - x^2}}
\]
\end{tcolorbox}

\end{frame}

%=====================================================================
\begin{frame}{Derivatives of Inverse Trigonometric Functions}

\begin{tcolorbox}[colframe=red!80!black, colback=white,
  title=\textbf{Derivatives of Inverse Trigonometric Functions}]
\[
\begin{aligned}
\frac{d}{dx} (\sin^{-1} x) &= \frac{1}{\sqrt{1 - x^2}} 
&\qquad
\frac{d}{dx} (\csc^{-1} x) &= -\frac{1}{x \sqrt{x^2 - 1}} \\[1em]
\frac{d}{dx} (\cos^{-1} x) &= -\frac{1}{\sqrt{1 - x^2}} 
&\qquad
\frac{d}{dx} (\sec^{-1} x) &= \frac{1}{x \sqrt{x^2 - 1}} \\[1em]
\frac{d}{dx} (\tan^{-1} x) &= \frac{1}{1 + x^2} 
&\qquad
\frac{d}{dx} (\cot^{-1} x) &= -\frac{1}{1 + x^2}
\end{aligned}
\]
\end{tcolorbox}

Try to prove all of the equations from the given properties of derivatives of inverse trigonometric functions.

\end{frame}
%=====================================================================
%=====================================================================

\end{document}